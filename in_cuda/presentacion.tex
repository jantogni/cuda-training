\documentclass{beamer}
\usepackage{graphics}
\usepackage{url}
\usepackage{ulem}
\usepackage{beamerthemesplit}
\usepackage{hyperref}
\usepackage{wrapfig}
\usepackage[spanish,activeacute]{babel}
\usepackage[utf8]{inputenc}
\usepackage{listings}
\usepackage{color}
\usetheme{Warsaw}
\usepackage{pstricks}
\lstset{breakatwhitespace=true,
language=C++,
columns=fullflexible,
keepspaces=true,
breaklines=true,
tabsize=3, 
showstringspaces=false,
extendedchars=true}


\title{Nividia Cuda}
\subtitle{Introducción y Ejemplos.}
\author{Jonathan Antognini C.}
\institute[]{Universidad Técnica Federico Santa María}
\date{\today}

\begin{document}
    \frame{\titlepage}
    \frame{\tableofcontents}
	\section{Introducción}
		\frame{
	\frametitle{Introducción}
	\begin{itemize}
		\item Compute Unified Device Architecture (CUDA), es una tecnología desarrollada por Nvidia Corporation en el 2007.
		\item Soportado de la serie G8X en adelante.
		\item Compilador (nvcc) + conjunto de herramientas de desarrollo en C/C++.
		\item Existen wrappers para otros lenguajes como: Python, Fortran, Java.
		\item El SDK está disponible para Linux, Windows y Mac.
	\end{itemize}
}

	\section{Nvidia Cuda}
		\frame{
	\centerline{Nvidia Cuda}
}

		\subsection{¿Por qué Cuda?}
			\frame{
	\frametitle{¿Por qué CUDA?}
	Se intenta explotar las ventajas de las GPU utilizando paralelismo soportado por los múltiples núcleos de una tarjeta gráfica.
	\\
	Las GPU Nvidia son arreglos de multiprocesadores, cada uno de los cuales tiene:
        \begin{itemize}
                \item Varios cores, que ejecutan el mismo programa concurrentemente.
                \item Memoria compartida, y mecanismo de sincronización.
        \end{itemize}
}

		\subsection{Empezando en cuda}
			\frame{
	\frametitle{Empezando en Cuda}
	Cuda Downloads: Última versión estable: 4.2. Hay un RC 5.0
	\url{http://developer.nvidia.com/cuda/cuda-downloads}
	\begin{itemize}
		\item Instalar toolkit.
		\item Instalar driver compatible.
		\item Instalar SDK.
	\end{itemize}

	Cuda library documentation:
	\url{http://www.clear.rice.edu/comp422/resources/cuda/html/index.html}
}

		\subsection{Programación Paralela en Cuda}
			\frame{
	\frametitle{Programación paralela en Cuda}
}

		\subsection{Cooperación entre threads}
			\frame{
	\frametitle{Cooperación entre Threads}
}

	\section{Ejemplos}
		\subsection{Suma de vectores}
			\begin{frame}[fragile]
	\frametitle{Ejemplos}
	\begin{itemize}
		\item Suma de vectores.
		\item Histograma.
	\end{itemize}
	%\begin{lstlisting}
	%	__global__ void add(int *a, int *b, int *c) {
	%	        int i = blockIdx.x * blockDim.x + threadIdx.x;
        %		if(i<N)
        %       		c[i] = a[i] + b[i];
	%	}
	%\end{lstlisting}
\end{frame}

		\subsection{Histograma}
			\frame{
	\frametitle{Histograma}
	%\centerline{EOF}
}

	\section{Conclusiones}
		\frame{
	\frametitle{Conclusiones}
	\begin{itemize}
		\item CUDA permite sacar provecho a las potencialidades de las GPU's actuales.
		\item Abordar problemas paralelos es más sencillo.
		\item Sin embargo es fácil programar en CUDA, pero es difícil conseguir rendimientos elevados.
	\end{itemize}
}

	\frame{
	\centerline{EOF}
}

\end{document}
